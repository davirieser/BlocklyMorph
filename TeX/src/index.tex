
\iffalse

  \begin{titlepage}
      \begin{center}
          \vspace*{1cm}

          \textbf{BlocklyMorph}

          \vspace{0.5cm}
          \vspace{1.5cm}

          \textbf{David Rieser}
      \end{center}
  \end{titlepage}

  \pagebreak

  \tableofcontents
  \pagebreak

\fi

\section{Communication Protocol}

Communication occurs over a multiplexed UART interface.\\
In the following, each side of a block that can be connected to, will be called "port".\\
The amount of ports of a block will be denoted as $P$.\\
The communication can be extended to arbritarly many ports.

\subsection{Pinout}

The following are the required pins on the microcontroller side:
\begin{itemize}
    \item UART RX and TX
    \item $\lceil ld(P)\rceil$ Pins for selecting which port is active
    \item $P$ Pins for Interrupts
\end{itemize}

The following are the required pins between the blocks:
\begin{itemize}
    \item VCC
    \item GND
    \item UART TX
    \item UART RX
\end{itemize}

\subsection{Communication Process}

For the following section the TX and RX of the microcontroller and block connection will be qualified by "uC" for the microcontroller and "bc" for the block connection respectively.\\
At any given time one of the ports will be selected (excluding during port selection transitions).\\
The selected port will have it's bc\_TX connected to uC\_RX and bc\_RX to uC\_TX thus enabling full-duplex communication.\\ \\
For the ports that are not selected the following configuration is used:\\
The bc\_RX line will be pulled to high to indicate to the other block that a block is connected.\\
At the same time a floating detector is used on the bc\_TX line to check if a block is connected.\\
The interrupt pin will be configured with a falling edge interrupt.\\
If you have data to send to the another block you set the given block to the active port and send a connection request frame.\\
Then you set a timer for 1ms during which you wait for a UART response frame.\\
If the other block does not respond before the timer runs out, you can move on to a different block or retry the current block.\\
(This should probably done with SYN -> ACK -> SYN-ACK where if the timer runs out you restart with SYN)\\
Here we have to make a case distinction:
\begin{itemize}
  \item If the other block has this port as the selected port:\\
    It will immidiately get the frame and communication can start immidiately.
  \item If the other block does not have this port as the selected port its RX will be connected to the interrupt pin.\\ 
    The UART start bit of the frame will pull the pin low and trigger a falling edge interrupt on the other block.\\
    If the other block is currently idle it will then change its selected port to the port where the frame was sent.\\
    If the other block is currently communicating with another block, it will finish this communication with this block and then reinitiate the communication with the block that sent the request.
  \item If the block is currently changing the selected port from this block to another block, it will miss the communication request.\\
    This will be caught by the timer reinitiating the request after some time.
\end{itemize}

\pagebreak

\subsection{Circuit Design}

I did create a circuit design on paper but was not able to convert it into LaTeX Code yet.\\
\\
\noindent First you need a 4-1 Demultiplexer for selecting which port to use (the port select signals) using two ports of the microcontroller OS1 and OS2 as the control signals and VCC as the input signal.\\
Then you need a transistor whose base is connected to uc\_TX the port select signal, collector to uc\_TX and emitter bc\_RX with a pull-up resistor.\\
A Demultiplexer input is connected to bc\_TX and it's control line is connected to the port select signal.\\
The first output of the Demultiplexer is connected to uc\_RX with a pull-up resistor.\\
Then you need to connect one 100k resistor from VCC to bc\_TX and one 100k resistor from bc\_TX to GND.\\
Then a window comparator is connected to bc\_TX with a window of 2.3V to 2.7V.\\
The second output of the Demultiplexer and the output of the window comparator is connected to the inputs of an AND-Gate and then connected to the interrupt pin.

\iffalse
\begin{figure}[!ht]
    \centering
    \begin{circuitikz}
    \end{circuitikz}

    \label{circ:UART_Multiplexing}
\end{figure}
\fi

\subsection{Problems}

\begin{itemize}
  \item How do we make sure that the window comparator does not oscillate during transmission? 
      If the selected board is disconnected during transmission we would like to know.\\
  \item How do we ensure that the RX and TX lines are crossed in the board connectors?
\end{itemize}

